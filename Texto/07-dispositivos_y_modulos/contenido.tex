\chapter{Dispositivos y Módulos}\LABCHAP{DISPS}
\pagestyle{esitscCD}

\section{Dispositivos}

Los dispositivos son elementos externos al sistema. El propio sistema está
diseñado para proveer servicios a estos dispositivos, por lo que podemos verlos
como un especie de usuario de la plataforma.

Cada dispositivo debe estar registrado en el sistema para poder interactuar con
él, por lo que deberán implementar un proceso en el cual puedan darse de alta en
la plataforma. Cada dispositivo estará asociado a un usuario, esto quiere decir
que a la hora de registrarse, el dispositivo debe proveer cierta información
para que la plataforma pueda determinar a qué usuario quedará asociado.

\subsection{Flotas}

La forma de asociar un dispositivo a un usuario es mediante las \textbf{flotas}.
Las flotas son agrupaciones independientes de dispositivos que puede crear el
usuario para gestionarlos. Cada usuario puede tener múltiples flotas. Al crear
una flota, se generará de forma automática una \texttt{fleet\_key}, que es una clave única
que representa a cada flota. Cuando un dispositivo ejecuta el proceso de
registro deberá proveer la clave para que el sistema lo reconozca como
perteneciente a su flota correspondiente a la hora de registrarlo en la base
de datos.

Dispositivos de diferentes flotas no pueden comunicarse entre ellos de forma
directa, es decir, si en una flota un dispositivo publica en el \emph{topic}
\texttt{temperature} y otro dispositivo de una flota diferente está suscrito
a este \emph{topic} no recibirá el mensaje. Sin embargo, siempre será posible
que una aplicación o dispositivo pertenezca a ambas flotas y sea capaz de
reenviar los mensajes.


\section{Módulos}

Los módulos son aplicaciones que simulan ser dispositivos. Su objetivo es actuar
como intermerdiario entre los dispositivos reales y la plataforma.

El dispositivo en el que corren los módulos se denominará hub. Una Raspberry Pi
puede ser muy adecuado como hub ya que puede implementar los protocolos
necesarios y puede comunicarse con los dispostivos reales usando un adaptador
Bluetooth o ZigBee.

Existen varios escenarios donde puede ser interesante usar módulos:

\subsection{Caso 1}
Los dispositivos no tienen capacidad de enviar mensajes \texttt{HTTP} o \texttt{MQTT}, que son los
protocolos usados por la plataforma. En tal caso, se puede crear un módulo que
corra en otro dispositivo capaz de implementar estos protocolos y, al mismo
tiempo, sea capaz de comunicarse con los dispositivos reales. Estos módulos
tienen la responsabilidad de registrarse como si fuesen dispositivos reales,
obtener datos y reenviarlos a los dispositivos correspondientes. Útiles para
dispositivos que usan \texttt{Bluetooth} o \texttt{ZigBee} como protocolo de comunicación.

\subsection{Caso 2}
Los dispositivos ya existen y no es posible modificar el software para que se
comunique con la plataforma directamente. Por ejemplo porque los dispositivos
los fabrica un tercero. En este caso, el módulo tiene una función de ``pasarela''
entre los dispositivos reales y la plataforma.

\section{Conexión al broker}

Para enviar datos a la plataforma será necesario disponer de los siguientes datos:

\begin{itemize}\itemsep1pt \parskip0pt \parsep0pt
\item \texttt{host}: El nombre o dirección IP del equipo donde está desplegada la plataforma.
\item \texttt{puerto}: El puerto donde se reciben los mensajes de los dispositivos mediante \texttt{MQTT}.
\item \texttt{user}: El \texttt{uuid} del dispositivo, generado en el proceso de registro.
\item \texttt{password}: La contraseña del dispositivo, generada en el proceso de registro.
\item \texttt{topics}: \texttt{<fleet\_key>/\#}. Donde ``\#'' es el nombre original del topic.
\end{itemize}

A la hora de publicar mensajes en un topic, el dispositivo debe preceder el
topic con el \texttt{fleet\_key}, por ejemplo, si se quiere publicar al topic
``temperature'', deberá enviarse el mensaje al topic \texttt{<fleet\_key>/temperature}.

Para suscribirse a un topic se aplicará la misma regla, precediendo el nombre
del topic original por el Fleet Key.

Se debe usar usar TLS ya que sino las credenciales viajarán en texto plano y
pueden ser capturadas por un tercero.
