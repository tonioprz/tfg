\chapter{Broker y autorización}\LABCHAP{DISPS}
\pagestyle{esitscCD}

\section{RabbitMQ}

Como \emph{broker} se usará \texttt{RabbitMQ}. El \emph{broker} es una de los
componentes más importantes de la plataforma, por no decir el más importante de
todos. Cada mensaje que llega al sistema desde un elemento externo, ya sea
un dispositivo o una aplicación del usuario debe hacerlo a través del \emph{broker}.
Además, los servicios internos de la plataforma también usan el \emph{broker} para
la comunicación.

\texttt{RabbitMQ} nos ofrece todo el sistema de colas de \texttt{AMQP} que ya
hemos descrito anteriormente, por lo que todos los datos de la plataforma
estarán en colas. Esto permite soportar picos de tráfico, ya que si los
servicios o las bases de datos no son capaces de soportar una carga elevada los
mensajes simplemente se encolarán y no serán descartados.

Además, gracias a unos plugins oficiales, es posible configurar a \texttt{RabbitMQ}
para que sea capaz de recibir mensajes \texttt{MQTT} y \texttt{STOMP}. Es importante
aclarar que en realidad no existe un \emph{broker} como tal para estos protocolos,
sino que realmente se hace una adaptación a \texttt{AMQP}, es decir, que aunque los
mensajes lleguen en formato \texttt{MQTT}, éstos se guardarán en colas \texttt{AMQP}
y los leerán servicios mediante este protocolo. Los servicios del núcleo de la
plataforma no implementan ni \texttt{MQTT} ni \texttt{STOMP}.

Un ejemplo de comunicación sería:

\begin{Verbatim}[frame=single]
DISPOSITIVO -> ADAPTADOR MQTT -> AMQP -> STREAM PROCESSOR -> RETHINKDB ->
-> STREAM PROCESSOR -> AMQP -> STOMP -> NAVEGADOR WEB
\end{Verbatim}

Por supuesto en sentido contrario también funcionaría.
