\chapter{API}\LABCHAP{API}
\pagestyle{esitscCD}

Lo primero y más importanta del sistema son los usuarios. Serán quienes exploten
las funcionalidades del sistema. Puesto que la finalidad de la plataforma es
establecer una comunicación entre los dispositivos de un cliente y su lógica,
tenemos dos roles:

\begin{itemize}\itemsep1pt \parskip0pt \parsep0pt
\item \indexit{Dispositivos:} Los componentes que obtienen información o
controlan elementos.
\item \indexit{Lógica:} El software del usuario que procesan la información
obtenida y controlan los dispositivos.
\end{itemize}

Para que el usuario pueda interactuar con la plataforma para conocer el estado
de los dispositivos se proveerá de una API RESTful que permita:

\begin{itemize}\itemsep1pt \parskip0pt \parsep0pt
\item \indexit{Registro de usuarios}
\item \indexit{Administración de flotas}
\item \indexit{Monitorización de dispositivos}
\end{itemize}

\section{Registro de usuario}

Para crear un usuario se enviará un mensaje HTTP a la API con el contenido del
código \ref{pet-01}:

\begin{minipage}{\textwidth}
\begin{lstlisting}[language=TeX,caption={Petición para creación de usuario}, breaklines=true, label=pet-01]
POST /users

{
  "email": "yo@dominio.com",
  "password": "secreto"
}
\end{lstlisting}
\end{minipage}

Una vez registrados, se puede hacer login y realizar peticiones autenticadas a
la API. Para realizar un \emph{login} debemos enviar el contenido del código \ref{pet-01}.

\begin{minipage}{\textwidth}
\begin{lstlisting}[language=TeX,caption={Petición de login}, breaklines=true, label=pet-02]
POST /users/login

{
  "email": "yo@dominio.com",
  "password": "secreto"
}
\end{lstlisting}
\end{minipage}

Como respuesta se obtiene el \texttt{acess\_token} y \texttt{user\_id}:

\begin{minipage}{\textwidth}
\begin{lstlisting}[language=TeX,caption={Respuesta a login}, breaklines=true, label=res-01]
{
  "id": "<access_token>",
  "ttl": 1209600,
  "created": "2013-12-20T21:10:20.377Z",
  "userId": "<user_id>"
}
\end{lstlisting}
\end{minipage}

A partir de ahora podemos realizar peticiones autorizadas enviando en la
cabezera del mensaje \texttt{HTTP} el \texttt{access\_token} el campo
  \texttt{Authorization}.

\section{Obtención de Fleet Key}

Para crear una flota es necesario haber realizado el proceso de registro y
disponer de un \texttt{user\_id} y \texttt{acess\_token}. Ser realizará la
petición indicada en el código \ref{pet-03}.

\begin{minipage}{\textwidth}
\begin{lstlisting}[language=TeX,caption={Creación de \texttt{fleet\_key}}, breaklines=true, label=pet-03]
POST /users/<user_id>/fleets

Authorization: <access_token>
\end{lstlisting}
\end{minipage}

Si todo ha ido correctamente se obtendrá como respuesta el mensaje de respuesta
\ref{res-02}

\begin{minipage}{\textwidth}
\begin{lstlisting}[language=TeX,caption={Obtencion de \texttt{fleet\_key}}, breaklines=true, label=res-02]
{
  "uuid": "<fleet_uuid>"
}
\end{lstlisting}
\end{minipage}

\section{Registro de dispositivos}

Cuando el dispositivo se inicie por primera vez, deberá realizar un proceso de
registro en el que usará la \texttt{fleet\_key} para poder obtener un
\texttt{uuid} y un \texttt{secret}. Para que un dispositivo pueda registrarse deberá ser capaz de comunicarse
mediante \texttt{HTTP}, en caso contrario, será necesario el uso de un dispostivo
intermedio capaz de realizar esta labor en nombre del dispositivo.

A este dispositivo intermedio lo llamaremos \texttt{hub} que puede ser cualquier
dispositivo que sea capaz de enviar mensajes \texttt{HTTP} y \texttt{MQTT} y
reenviarlos en el protocolo que sea a los dispositivos reales. Podría ser, por
ejemplo, un \emph{router} o una \emph{Raspberry Pi}.

Un caso podría ser un \emph{Arduino} que se comunicase mediante un módulo \texttt{NRF24} con
un programa que se ejecute en una \emph{Raspberry Pi}. Será dicho programa de la
\emph{Raspberry Pi} el responsable de realizar el registro, de forma que el dispositivo no
tiene porqué tener conocimiento del proceso ni de la plataforma. Cabe destacar
que, en este caso, será la aplicación en cuestión quien debe conocer la \texttt{fleet\_key}.

Esto puede ser útil en caso de tener dispositivos que no podemos adaptar al
sistema, bastaría con crear este intermediario para poder usar el dispositivo
con la plataforma.

En cualquier caso, el procedimiento será enviar un mensaje \texttt{HTTP} a la
API con el contenido de la petición \ref{pet-04}.

\begin{minipage}{\textwidth}
\begin{lstlisting}[language=TeX,caption={Registro de dispositivo}, breaklines=true, label=pet-04]
POST /fleets/<fleety_key>/registerMote
\end{lstlisting}
\end{minipage}

\begin{minipage}{\textwidth}
\begin{lstlisting}[language=TeX,caption={Respuesta de registro de dispositivo}, breaklines=true, label=res-03]
{
  "uuid":   "<Nuevo UUID de dispositivo>",
  "secret": "<Nuevo secreto>"
}
\end{lstlisting}
\end{minipage}

Una vez obtenidas las credenciales para poder enviar datos deberían persistirse
para futuros usos.
