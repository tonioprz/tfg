% !TEX root =../main.tex
\chapter*{Resumen}
\pagestyle{especial}
\chaptermark{Resumen}
\phantomsection
\addcontentsline{toc}{listasf}{Resumen}

\lettrine[lraise=-0.1, lines=2, loversize=0.2]{E}n el laboratorio de Automática
hay robots de la marca ABB destinados a prácticas para alumnos. Durante estas 
prácticas es común la invasión del espacio de trabajo de los robots. Para reducir 
la exposición a riesgos asociados a estos trabajos se introducen cintas transportadoras
que permiten el movimiento de piezas sin peligro para los alumnos. En este trabajo
se desarrolla un dispostivo que permite el uso de dicha cinta de forma segura, 
además de dar información relevante sobre la posición de las piezas al robot.
El sistema está compuesto de un Arduino que posiciona las piezas en los ejes de la cinta
mediante un encoder y un calibre digital. También cuenta con una placa electrónica que
centraliza todas las conexiones y permite el funcionamiento en los distintos modos establecidos.
Además, el sistema está pensado para ser reparable, usando piezas existentes en el 
mercado para sustituirlas en caso de ser necesario.



\chapter*{Abstract}
\pagestyle{especial}
\chaptermark{Abstract}
\phantomsection
\addcontentsline{toc}{listasf}{Abstract}

\lettrine[lraise=-0.1, lines=2, loversize=0.2]{I}{n} the lab of automatics