% !TEX root =../main.tex
\chapter*{Resumen}
\pagestyle{especial}
\chaptermark{Resumen}
\phantomsection
\addcontentsline{toc}{listasf}{Resumen}

\lettrine[lraise=-0.1, lines=2, loversize=0.2]{E}n el laboratorio de Automática
de la Escuela Técnica Superior de Ingeniería de la Universidad de Sevilla 
hay robots de la marca ABB destinados a prácticas para alumnos. Durante estas 
prácticas es común la invasión del espacio de trabajo de los robots. El objetivo de
este proyecto es reducir 
la exposición a riesgos asociados a estos trabajos se introducen cintas transportadoras
que permiten el movimiento de piezas sin peligro para los alumnos. Para ello,
se desarrolla un dispositivo que permite el uso de dicha cinta de forma segura, 
además de dar información relevante sobre la posición de las piezas al robot.
El sistema está compuesto de un Arduino que posiciona las piezas en los ejes de la cinta
mediante un encoder y un calibre digital. También cuenta con una placa electrónica que
centraliza todas las conexiones y permite el funcionamiento para los distintos modos establecidos
según las necesidades de los alumnos.
Además, el sistema está pensado para ser fácilmente reparable, empleando piezas 
estandarizadas existentes en el mercado para sustituirlas en caso de ser necesario.
De esta forma se pone solución a un problema habitual de los laboratorios de prácticas con brazos
robóticos haciendo de este lugar un lugar más seguro para estudiantes y profesores.


\chapter*{Abstract}
\pagestyle{especial}
\chaptermark{Abstract}
\phantomsection
\addcontentsline{toc}{listasf}{Abstract}

\lettrine[lraise=-0.1, lines=2, loversize=0.2]{T}here are ABB robots in the Automation laboratory of 
the School of Engineering of the University of Sevilla for student training; during these activities, 
it is frequent for the robots to invade the workspace. 
Therefore, the aim of this project has been to reduce risks associated with these activities through the 
introduction of conveyor belts that allow the different parts to be moved 
substantially reducing the risk to the students. For this purpose, a system that allows the use of the belt 
in a safer way has been developed. This system provides information regarding the position of the 
pieces to the robot. This system consists of an Arduino that positions the parts on the 
axes of the belt by using an encoder and a digital caliper. The system is 
equipped with an electronic board that centralizes the connections and allows the system to function in 
different modes defined by the students. Furthermore, the system is designed to be easily repairable, as 
it uses standardized available parts. Thus, a frequent problem in practice laboratories with robotic arms 
has been solved, thereby improving the safety of students and teachers.