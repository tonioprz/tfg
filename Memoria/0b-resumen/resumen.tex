% !TEX root =../main.tex
\chapter*{Resumen}
\pagestyle{especial}
\chaptermark{Resumen}
\phantomsection
\addcontentsline{toc}{listasf}{Resumen}

\lettrine[lraise=-0.1, lines=2, loversize=0.2]{E}n el laboratorio de Automática
de la Escuela Técnica Superior de Ingeniería de la Universidad de Sevilla 
hay robots de la marca ABB destinados a prácticas para alumnos. Durante estas 
prácticas es común la invasión del espacio de trabajo de los robots. El objetivo de
este proyecto es reducir 
la exposición a riesgos asociados a estos trabajos se introducen cintas transportadoras
que permiten el movimiento de piezas sin peligro para los alumnos. Para ello,
se desarrolla un dispositivo que permite el uso de dicha cinta de forma segura, 
además de dar información relevante sobre la posición de las piezas al robot.
El sistema está compuesto de un Arduino que posiciona las piezas en los ejes de la cinta
mediante un encoder y un calibre digital. También cuenta con una placa electrónica que
centraliza todas las conexiones y permite el funcionamiento para los distintos modos establecidos
según las necesidades de los alumnos.
Además, el sistema está pensado para ser fácilmente reparable, empleando piezas 
estandarizadas existentes en el mercado para sustituirlas en caso de ser necesario.
De esta forma se pone solución a un problema habitual de los laboratorios de prácticas con brazos
robóticos haciendo de este lugar un lugar más seguro para estudiantes y profesores.


\chapter*{Abstract}
\pagestyle{especial}
\chaptermark{Abstract}
\phantomsection
\addcontentsline{toc}{listasf}{Abstract}

\lettrine[lraise=-0.1, lines=2, loversize=0.2]{I}{n} the lab of automatics