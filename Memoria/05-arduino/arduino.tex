\chapter{Desarrollo en Arduino}\label{chp-05}

\lettrine[lraise=-0.1, lines=2, loversize=0.2]{U}na parte imprescindible del proyecto es la programación del propio Arduino, que es el que centraliza todo el control del sistema. La parte relacionada con el control del motor y la programación del encoder está bien documentada en el trabajo Tapia\cite{tapia}, mientras que para la medición del calibre se ha partido de la web \cite{caliper} al igual que en el diseño electrónico ya comentado en el capítulo \ref{chp-03}.

\section{Diagrama de flujo}

\begin{figure}[hbtp]
    \centering
    \includegraphics[height=0.5\textheight]{05-arduino/diagramaflujo.pdf}
    \caption{Diagrama de flujo del programa}
    \label{fig:diagramaflujo}
    \end{figure}

En el diagrama de la figura \ref{fig:diagramaflujo} se observa los distintos estados por los que pasa el 
programa. En primer lugar, se realiza una comprobación de los estados de microcontrolador y local para 
configurar los pines como corresponde. En caso de que no haya microcontrolador, se configurarán los pines
IN3, IN4 y ENB como entradas y el dispositivo esperará a que se salga de dicho modo, manteniéndose en estado 
"f". Si se configura como remoto, el Arduino entrará en estado "r" (remoto) y comprobará continuamente si se
recibe una orden por parte del controlador del ABB para realizar dicho movimiento.

El caso que tiene más estados intermedios es el modo local, ya que requiere varios menús dentro de la pantalla
para la interacción con el usuario. En este modo de funcionamiento se avanza en el menú con el botón enter y se
retrocede con el botón escape. Tras realizar las configuraciones del sistema, el sistema entrará en estado 
"i" (inicial), mostrado un mensaje de bienvenida en el LCD y esperando al botón enter para avanzar de menú. Una 
vez pulsado enter, se pasa al modo selección de avance absoluto o discreto, estado "s". En este estado, mediante
las flechas de selección se cambia entre avance discreto y avance absoluto. Una vez avanzado al siguiente menú, 
entramos en el modo selección de distancia de movimiento "d". En este modo, mediante las flechas se irá incrementando
o disminuyendo la posición final deseada. Por último, se tomará toda la información recibida por el usuario para 
pasar al estado de motor en movimiento "m".

Independientemente si se accede al estado "m" mediante modo local o remoto, la cinta se moverá hasta la posición 
deseada mediante el mismo algoritmo. Primero se aproxima la pieza a la posición final avanzando a velocidad constante
para posicionar finalmente la pieza mediante un PID. Una vez terminada la operación, vuelve al modo inicial para 
seguir esperando órdenes.

La seta de emergencia puede ser pulsada en cualquier instante en todos los modos de funcionamiento, la cinta parará 
automáticamente y se entrará en estado emergencia "e". 
Se trata de un sistema de seguridad independiente y una vez activado este modo no se podrá realizar ninguna acción 
hasta que se desarme la seta de emergencia.

\section{Programa final}

A continuación se muestra el programa final. Este se divide en dos archivos, "main.h" y "main.ino". El programa puede ser 
cargado a la placa mediante el Arduino IDE que se puede descargar desde la págica oficial \cite{arduino}. Además, en la propia
página se puede acceder a la documentación referente a las funciones utilizadas en el programa.

Por otro lado, dos librerías han sido utilizadas para el funcionamiento del LCD y de la conexión por Ethernet. Para el caso
de la librería para el LCD se ha utilizado la librería \cite{lcd}, ya que la conexión se realiza mediante protocolo I2C.
Este protocolo permite que un segundo microcontrolador integrado dentro de la placa del LCD sea el que lo controle, evitando
tener que realizar todas las conexiones que requiere el mismo directamente hacia el Arduino y delegando el control al mismo.
Para la conexión mediante Ethernet se ha utilizado la librería \cite{ethernet}. Este caso es similar al anterior, en el cual
mediante el módulo Arduino Ethernet Shield acoplado al Arduino Mega se realiza una comunicación SPI con el microcontrolador 
W5100 del módulo, que es el que realiza todas las operaciones necesarias para permitir la conexión.

El archivo "main.h", mostrado a continuación en el código \ref{main_h} es el encargado de definir todas las variables globales
y definir macros para los pines. Este archivo se separa del principal para tener un acceso más rápido a las variables por si 
se necesita realizar un ajuste y que el código principal sea más fácil de leer. En éste se incluyen los parámetros del PID,
la inicialización del estado, la configuración inicial de la poisición y el desplazamiento, además de ciertas variables auxiliares
que se utilizan a lo largo de todo el programa.


\begin{lstlisting}[language=,caption={Declaración de variables, "main.h"}, breaklines=true, label=main_h]
// Definición de pines
#define faseA 2
#define faseB 3

#define BUTTON_ESC 22
#define BUTTON_ENTER 23
#define BUTTON_UP 24
#define BUTTON_DOWN 25

#define BUTTON_LOCAL 28
#define BUTTON_MICRO 29
#define BUTTON_EMERGENCIA 30

#define SENSOR_FOTO 31

#define CAL_CLK 32
#define CAL_DATA 33

#define ENB 5
#define IN3 40
#define IN4 41

// Declaración de objetos de LCD y relacionados con Ethernet
LiquidCrystal_I2C lcd(0x27,16,2);
byte MAC[] = { 0xDE, 0xAD, 0xBE, 0xEF, 0xFE, 0xED }; // Dirección MAC del dispositivo
EthernetServer servidor(4012); // Puerto donde se transmite la información

// Declaración de variables booleanas auxiliares
bool enter;
bool esc;
bool up;
bool down;
bool esc_ant = 0;
bool enter_ant = 0;
bool up_ant = 0;
bool down_ant = 0;

bool emergencia;
bool local;
bool micro;
bool fotoelectrico;

bool emergencia_ant;
bool local_ant;
bool micro_ant;

// Declaración e inicialización de variable de posición 'x' e 'y'
long posicion = 0;
long * pposicion = &posicion;
float posy = 0;

// Inicialización de variables de estado y desplazamiento
char estado = 'i';
bool discreto = 0;
long desplazamiento = 1000;
long objetivo = 0;

// Definición de parámetros de PID
float kp = 0.5;
float Ti = 2;
float Td = 0.1;
float T = 0.01;
    
\end{lstlisting}


Por último, se incluye el código \ref{main_cpp} que es el código el archivo "main.ino". Este archivo contiene el programa en sí
con todas sus funciones y protocolos. Éste cuenta con comentarios relevantes que explican el código que se muestra.

\input{05-arduino/maincpp.tex}