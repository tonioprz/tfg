% !TEX root =../LibroTipoETSI.tex
%El anterior comando permite compilar este documento llamando al documento raíz
\chapter{Introducción}\label{chp-01}

\section{Motivación}

\lettrine[lraise=-0.1, lines=2, loversize=0.2]{H}oy en día es innegable que
las tecnologías enfocadas al IoT (Internet de las
cosas) están en pleno auge. La tendecia es conectar todo lo que se pueda a
Internet para así hacerlo inteligente.

Imaginemos que un fabricante decide crear un dispositivo, digamos que desarrolla
un sensor de temperatura para que un usuario pueda monitorizar la temperatura
de una habitación mediante una interfaz web o app móvil. Al poco de plantear
alguna solución encontrará que desarrollar toda una arquitectura para poder
comunicar su sensor de temperatura con la aplicación web puede ser un proceso
complejo. Será necesario obtener datos de miles de sensores, tratar los datos,
almacenarlos, poder obtenerlos desde la aplicación web o móvil, etc.

Mientras que una organización de gran tamaño con suficientes recursos puede
abordar el problema, para pequeñas compañías o startups puede suponer un
handicap. El desarrollo de un backend sobre el que se apoye su producto puede
ser una tarea que termine por hacer que el proyecto sea inviable, ya que no
permite al desarrollador centrar todos sus esfuerzos en desarrollar su producto
y le obliga a gastar recursos en construir y mantener su backend.

En el panorama actual existe una gran alternativas a la hora de elegir las
diferentes tecnologías que compondrán el sistema. El mero hecho de realizar
un estado del arte ya supone un esfuerzo. Para solucionar cada pequeño problema
podemos encontrar una gran variedad de soluciones y a la hora de la integración
de las diferentes partes pueden surgir más problemas.

Todo esto no hace más que suponer una barrera para los fabricantes que puede
desembocar en que el proyecto nunca sea llevado a cabo.

\section{Escenario}

\subsection{Usuario}

Llamaremos usuario a la persona, personas u organización que utiliza la plataforma
que vamos a diseñar para su beneficio. Queremos cubrir una necesidad, que en
este caso es comunicar dos elementos: los dispositivos integrados del usuario y
la lógica de negocio del usuario.

Por supuesto, esto deberá de hacerse de forma totalmente transparente y de la
forma más eficiente posible. El usuario no tiene por qué saber cómo funciona
el sistema de forma interna, sólamente deberá saber cómo interactuar con él.

\subsection{Dispositivos integrados}

En primer lugar tenemos los dispositivos del usuario, que son los
elementos que queremos dotar de conectividad para que puedan comunicarse de
forma eficiente y robusta. Debemos tener en cuenta que cuando hablamos de
sistemas integrados no podemos suponer que contamos con los mismo recursos
que en un equipo estándar. La cantidad de memoria, de almacenamiento, de
ancho de banda o incluso de energía pueden ser muy limitadas, por lo que no
podemos aplicar las mismas soluciones que aplicaríamos en un entorno donde
nos sobran dichos recursos.

Con esto se quiere decir que protocolos como \texttt{HTTP} que funcionan de
forma correcta en un equipo actual no tienen por qué funcionar igual de bien
en un dispositivo que cuenta con poca memoria.
Si en lugar de \emph{WiFi} se usa un protocolo más simple orientado a eficiencia
energética, \texttt{HTTP} puede ser muy pesado y su implementación puede consumir
mucha memoria, energía o ancho de banda.

Tendremos que buscar soluciones acordes a esta tecnología.

\subsection{Lógica de negocio}

Es el conjunto de \emph{software} que desea comunicarse con los dispositivos, ya
sea para recolectar los datos que unos sensores captan o para accionar unos actuadores.
Toda esta lógica pertenece al usuario y puede ser desde un pequeño \emph{script}
en \emph{Python} hasta una compleja infraestructura desplegada en \emph{Amazon}.
Puesto que tiene otras características diferentes a los dispositivos tendrán otros
requisitos diferentes, por lo que la forma en la que interaccionarán con la plataforma
no tiene por qué ser la misma que los dispositivos.

Uno de los requisitos para la lógica de negocio es que debe poder funcionar en
un navegador web, de forma que se podría crear una aplicación web que se comunique
\textbf{directamente} con la plataforma.
