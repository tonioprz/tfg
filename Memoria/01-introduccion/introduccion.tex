!TEX root =../LibroTipoETSI.tex
\chapter{Introducción}\label{chp-01}


\lettrine[lraise=-0.1, lines=2, loversize=0.2]{E}{n} las prácticas de laboratorio realizadas
por parte de alumnos durante la docencia de los cursos de Robótica del Departamento de Ingeniería
de Sistemas y Automática de la Escuela Técnica Superior de Ingeniería de Sevilla, los alumnos
deben programar el movimiento de un brazo robótico presente en dicho laboratorio. El objetivo
es que una pieza sea trasladada por el robot desde un punto de la mesa de trabajo hacia una
segunda posición. El proceso se realiza colocando manualmente la pieza, con los problemas que
ello conlleva. Por un lado, la precisión es cuestionable, ya que el propio alumno no tiene una
referencia sobre la cual poder repetir el proceso de forma eficaz y el error introducido al sistema
es alto. Por otro lado, al invadir el espacio de trabajo del brazo robótico continuamente se
producen riesgos innecesarios impropios de las normas de seguridad en la industria.

Este trabajo es la continuación de \cite{tapia} y \cite{rea}, que sentaron las bases del proyecto.
En esta ocasión, el enfoque es en la implementación sobre los equipos del laboratorio. Por ello,
el sistema creado debe quedar en una caja donde se realicen las conexiones electrónicas y todos
los dispositivos. Además el sistema debe tener componentes fácilmente sustituibles para facilitar
las reparaciones.

El sistema cuenta con las siguientes características:
\begin{itemize}
	\item Posicionamiento de piezas en medidas de ejes X e Y que el usuario requiera para 
	interactuar con el robot.
	\item Conexión entre Arduino y RobotStudio mediante protocolo TCP/IP para comunicaciones.
	\item Funcionamiento sin Arduino mediante señales digitales del robot.
	\item Funcionamiento sin conexión directa entre RobotStudio y Arduino.
\end{itemize}

\section{Modos de funcionamiento}\label{sec-00}



%\subsection{Compilación}
