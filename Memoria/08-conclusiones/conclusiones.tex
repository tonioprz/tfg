\chapter{Conclusiones}\label{chp-08}

Tras el desarrollo por completo, teórico y práctico del proyecto, 
se puede llegar a la conclusión de que el sistema construido cumple 
el objetivo principal, otorga una solución de seguridad para los 
laboratorios de prácticas del departamento de Sistemas y Automática. 
Es un aparato que los alumnos pueden manejar de manera sencilla a 
través de la interfaz que se ha dispuesto en la tapa. 

Se ha puesto en funcionamiento el dispositivo, desarrollando perfectamente 
todas las funcionalidades de los modos pensados para la comodidad de los alumnos. 

Además de cumplir el objetivo principal, se han logrado alcanzar ciertos 
objetivos secundarios propuestos inicialmente como la accesibilidad para 
poder realizar reparaciones fáciles y rápidas. Su fabricación empleando 
elementos electrónicos estandarizados como placas de Arduino, fuentes de 
alimentación o controladores de motores genéricos ha allanado el camino de la
construcción del dispositivo y facilitado su reparabilidad ya que sus
elementos se encuentran en el mercado con facilidad.

Todo esto ha influido en el presupuesto de construcción del proyecto con un 
impacto reducido, por lo que económicamente se ha obtenido un resultado 
aceptable con bajo presupuesto y con bajos recursos, empleando una impresora
3D y ahorrando en materiales de esta forma.

Es conocido que los residuos electrónicos son un grave problema en la
actualidad, por ello, el hecho de realizar el proyecto con elementos
reciclables y con fácil reparación, prolonga la vida útil del equipo, 
así como facilita su reciclaje cuando su utilización no resulte más provechosa. 
La placa de Arduino es reprogramable, los componentes electrónicos como 
botones e interruptores pueden ser reutilizados y la caja de plástico 
puede ser reciclada o reutilizada. Por ello el impacto medioambiental 
del proyecto es ínfimo, buscando una solución a un problema actual con 
pocos recursos y pocos residuos.

Como desarrollo futuro, una propuesta interesante sería realizar un sistema 
integrado, reduciendo considerablemente el número de componentes y de conexiones
aéreas realizadas. Para ello, sería necesario integrar todo en una única placa 
con un microcontrolador específico y toda la electrónica necesaria, que sería 
notablemente inferior en cuanto al número de componentes para poder desarrollarlo.