\begin{lstlisting}[language=,caption={Ejemplo de programa con microcontrolador}, breaklines=true, label=cod_micro]
MODULE Module1  
    !Declaración de variables del programa
    !Objeto de socket de conexión
    VAR socketdev my_socket;

    !Variable de estado
    VAR string estado:=0;

    !Cadenas de caracteres de recepción
    VAR rawbytes receive_string;
    VAR string string1;
    VAR string posx_str;
    VAR string posy_str;
    VAR string fotoele_str;

    !Posiciones de los datos a lo largo de la cadena recibida
    VAR num xpos;
    VAR num ypos;
    VAR num fepos;
    
    !Valores numéricos finales recibidos
    VAR num posx;
    VAR num posy;
    VAR byte fotoele;
    
    !Comprobación de recepción correcta
    VAR bool okposx:=true;
    VAR bool okposy:=true;

    !Bucle principal del programa
    PROC main()
        !Se cierran los posibles sockets abiertos
        SocketClose my_socket;
        
        WaitTime 0.2;
        
        !Función de lectura de posición y estado
        leer;

        !Movimiento absoluto
        mov_abs(200);

        !Movimiento discreto
        mov_dis(100);

        WaitTime 0.5;
    ENDPROC
    
    !Función de apertura de socket
    PROC abricomunicacion()        
        SocketCreate my_socket; !crea el socket
        SocketConnect my_socket, "192.168.50.200", 4012;        
    ENDPROC
    
    !Función de lectura de estado y 
    PROC leer()
        !Se abre el socket y conecta al Arduino
        abricomunicacion;
        
        !Escribe en el socket la cadena "STATUS" para que el Arduino envíe sus datos
        SocketSend my_socket,\Str:="STATUS"; 
        WaitTime 0.1; !espera un tiempo

        !Recibe la respuesta 
        ClearRawBytes receive_string;
        SocketReceive my_socket \RawData := receive_string,\Time:=WAIT_MAX;
        
        !Desempaqueta los bytes y los convierte en una cadena de caracteres
        UnpackRawBytes receive_string, 1, string1 \ASCII:=32;
        
        !Se buscan las posiciones de cada variable a lo largo de la cadena
        xpos        := StrFind(string1, 1, "=");
        ypos        := StrFind(string1, xpos+1, "=");
        fepos       := StrFind(string1, ypos+1, "=");
        
        !Se trocea la cadena para obtener subcadenas con los datos recibidos
        estado      := StrPart(string1, 0, 1);
        posx_str    := StrPart(string1, xpos+1, ypos - xpos - 3);
        posy_str    := StrPart(string1, ypos+1, fepos - ypos - 3);
        fotoele_str := StrPart(string1, fepos+1, 1);

        !Se transforman las cadenas a los tipos de datos que les corresponden
        okposx      :=  StrToVal(posx_str,posx);
        okposy      :=  StrToVal(posy_str,posy);
        fotoele     :=  StrToByte(fotoele_str);
        
        !Conversión de pulsos a milímetros
        posx := posx / 100;

        !Se cierra el socket
        ClearRawBytes receive_string;
        SocketClose my_socket;
    ENDPROC

    !Funciones de movimiento. Se envía una cadena con un caracter y la distancia a recorrer. Si el carácter es "M", el movimiento es absoluto, mientras que si es "R" es relativo.
    PROC mov_abs(num distancia)
        abricomunicacion;
        SocketSend my_socket,\Str:="M;X=" + ValToStr(distancia*100) + ";";
        WaitTime 0.1;
        SocketClose my_socket;
    ENDPROC
    
    PROC mov_dis(num distancia)
        abricomunicacion;
        SocketSend my_socket,\Str:="R;X=" + ValToStr(distancia*100) + ";";
        WaitTime 0.1;
        SocketClose my_socket;
    ENDPROC
ENDMODULE
\end{lstlisting}
    